\documentclass{llncs}

\usepackage[a4paper,includeheadfoot,top=1.65in,bottom=1.65in,left=1.73in,hcentering]{geometry}
\usepackage[pdftex]{graphicx}
\usepackage{amsmath}
\usepackage{booktabs}
\usepackage{float}
\restylefloat{table}

% Ezek egyik\'{e}nek bekapcsol\'{a}s\'{a}val a magyar \'{e}kezetes karakterek
% k\"{o}zvetlen\"{u}l is felhaszn\'{a}lhat\'{o}k:
\usepackage[latin2]{inputenc} 
%\usepackage[utf8]{inputenc}

\usepackage[hungarian]{babel}
\selectlanguage{hungarian} 

\begin{document}

\pagestyle{myheadings}
\def\leftmark{{\rm IV. Magyar Sz\'am\'\i t\'og\'epes Nyelv\'eszeti Konferencia}}
\def\rightmark{{\rm Szeged, 2015}}

\setcounter{page}{3}

\title{\ \break Lexik\'{a}lis behelyettes\'{i}t\'{e}s magyarul}
\author{Tak\'{a}cs D\'{a}vid\inst{1}, G\'{a}bor Kata\inst{2}}
\institute{1 PREZI, e-mail: takdavid@gmail.com\\2 INRIA, e-mail: kata.gabor@inria.fr}

\maketitle

\begin{abstract}
Cikk\"{u}nkben a lexik\'{a}lis behelyettes\'{i}t\'{e}si feladat (lexical substitution) magyarra adatpt\'{a}l\'{a}s\'{a}t \'{e}s k\'{e}t k\"{u}l\"{o}nb\"{o}z\H{o} megold\'{a}s\'{a}nak tesztel\'{e}s\'{e}t t\'{a}rgyaljuk. A lexik\'{a}lis behelyettes\'{i}t\'{e}s c\'{e}lja olyan algoritmus megalkot\'{a}sa, mely k\'{e}pes egy lexik\'{a}lis egys\'{e}g egy-egy mondatbeli el\H{o}fordul\'{a}s\'{a}t m\'{a}sik egys\'{e}ggel helyettes\'{i}teni olyan m\'{o}don, hogy a mondat eredeti jelent\'{e}s\'{e}t a lehet\H{o} legjobban meg\H{o}r\'{i}zze. A feladat \'{a}ltalunk kipr\'{o}b\'{a}lt v\'{a}ltozat\'{a}ban az algoritmusnak kell elv\'{e}geznie a behelyettes\'{i}t\'{e}sre javasolt jel\"{o}ltek gener\'{a}l\'{a}s\'{a}t, valamint a sz\"{o}vegk\"{o}rnyezetbe legjobban illeszked\H{o} lexik\'{a}lis egys\'{e}g kiv\'{a}laszt\'{a}s\'{a}t. A ki\'{e}rt\'{e}kel\'{e}s sor\'{a}n a rendszer \'{a}ltal javasolt jel\"{o}lteket annot\'{a}torok \'{a}ltal adott v\'{a}laszokkal vetj\"{u}k \"{o}ssze. A behelyettes\'{i}t\'{e}si feladat magyarra alkalmaz\'{a}s\'{a}nak c\'{e}lja, hogy felm\'{e}rj\"{u}k a disztrib\'{u}ci\'{o}s szemantikai m\'{o}dszerek m\H{u}k\"{o}d\'{e}s\'{e}nek hat\'{e}konys\'{a}g\'{a}t, valamint - a m\'{a}s nyelveken v\'{e}gzett k\'{i}s\'{e}rletekkel \"{o}sszevetve - k\'{e}pet kapjunk az esetlegesen felmer\"{u}l\H{o} magyar-specifikus kih\'{i}v\'{a}sokr\'{o}l: a rendelkez\'{e}sre \'{a}ll\'{o} er\H{o}forr\'{a}sokr\'{o}l, illetve a nyelvi jellegzetess\'{e}gekb\H{o}l ad\'{o}d\'{o} probl\'{e}m\'{a}kr\'{o}l.
\\[2mm]
{\bf Kulcsszavak:} lexik\'{a}lis behelyettes\'{i}t\'{e}s, lexik\'{a}lis szemantika, disztrib\'{u}ci\'{o}s szemantika
\end{abstract}

%
\section{Bevezet\'es}
%

A lexik\'{a}lis szemantikai kutat\'{a}sok, ezen bel\"{u}l a disztrib\'{u}ci\'{o}s szemantika egyre nagyobb teret nyer a sz\'{a}m\'{i}t\'{o}g\'{e}pes nyelv\'{e}szet k\"{u}l\"{o}nb\"{o}z\"{o} \'{a}gaiban (pl. szinonima-detekt\'{a}l\'{a}s, szemantikai rel\'{a}ci\'{o}k tanul\'{a}sa, ontol\'{o}gi\'{a}k/lexikai adatb\'{a}zisok automatikus \'{e}p\'{i}t\'{e}se, dokumentum-kategoriz\'{a}l\'{a}s). A korpuszb\'{o}l kinyert vektori\'{a}lis reprezent\'{a}ci\'{o}k ki\'{e}rt\'{e}kel\'{e}s\'{e}nek egyik lehets\'{e}ges m\'{o}dja az eredm\'{e}nyek interg\'{a}l\'{a}sa valamilyen nyelvtechnol\'{o}giai alkalmaz\'{a}sba, \'{a}m erre nem minden esetben ny\'{i}lik k\"{o}zvetlen lehet\H{o}s\'{e}g. Ennek megfelel\H{o}en egyre t\"{o}bbf\'{e}le ki\'{e}rt\'{e}kel\'{e}si feladat \'{e}s gold standard l\'{e}tezik a t\'{e}m\'{a}ban (l. SemEval kamp\'{a}nyok). A vektoros szemantikai reprezent\'{a}ci\'{o}k lehet\H{o}v\'{e} teszik, hogy a szavak jelent\'{e}se/szemantikai tartalma k\"{o}z\"{o}tti hasonl\'{o}s\'{a}got, vagy \'{e}ppen a szisztematikus elt\'{e}r\'{e}seket sz\'{a}mszer\H{u}s\'{i}ts\"{u}k. Egyes ki\'{e}rt\'{e}kel\'{e}si szabv\'{a}nyok az annot\'{a}torok \'{a}ltal megadott (szint\'{e}n numerikus) szemantikai hasonl\'{o}s\'{a}gi \'{e}rt\'{e}keket \cite{rubens65} vagy plauzibilit\'{a}si \'{i}t\'{e}leteket \cite{pado07} haszn\'{a}lnak. A lexik\'{a}lis behelyettes\'{i}t\'{e}s el\H{o}nye az el\H{o}bbi ki\'{e}rt\'{e}kel\'{e}si m\'{o}dszerekkel szemben, hogy az annot\'{a}torok sz\'{a}m\'{a}ra term\'{e}szetesebb, a nyelvi tud\'{a}st k\"{o}zvetlenebb\"{u}l mozg\'{o}s\'{i}t\'{o} feladatot jelent, \'{e}s nem t\'{a}maszkodik el\H{o}re meghat\'{a}rozott jelent\'{e}st\'{a}rakra vagy nyelv\'{e}szeti defin\'{i}ci\'{o}kra (szemben p\'{e}ld\'{a}ul a hagyom\'{a}nyos WSD feladattal).\\

A lexik\'{a}lis behelyettes\'{i}t\'{e}s \cite{mccarthynavigli10,fabre14} c\'{e}lja olyan algoritmus megalkot\'{a}sa, mely k\'{e}pes egy lexik\'{a}lis egys\'{e}g (egyszer\H{u} sz\'{o}, t\"{o}bbszavas kifejez\'{e}s) egy-egy mondatbeli el\H{o}fordul\'{a}s\'{a}t m\'{a}sik egys\'{e}ggel helyettes\'{i}teni olyan m\'{o}don, hogy a mondat eredeti jelent\'{e}s\'{e}t a lehet\H{o} legjobban meg\H{o}rizze. A feladat \'{a}ltalunk kipr\'{o}b\'{a}lt v\'{a}ltozat\'{a}ban az algoritmusnak kell elv\'{e}geznie a behelyettes\'{i}t\'{e}sre javasolt jel\"{o}ltek (els\H{o}sorban, de nem kiz\'{a}r\'{o}lag szinonim\'{a}k) gener\'{a}l\'{a}s\'{a}t, valamint a sz\"{o}vegk\"{o}rnyezetbe legjobban illeszked\H{o} lexik\'{a}lis egys\'{e}g kiv\'{a}laszt\'{a}s\'{a}t. A ki\'{e}rt\'{e}kel\'{e}s sor\'{a}n a rendszer \'{a}ltal javasolt jel\"{o}lteket annot\'{a}torok \'{a}ltal adott v\'{a}laszokkal vetj\"{u}k \"{o}ssze. A behelyettes\'{i}t\'{e}si feladat magyarra alkalmaz\'{a}s\'{a}nak c\'{e}lja, hogy felm\'{e}rj\"{u}k a lexik\'{a}lis/disztrib\'{u}ci\'{o}s szemantikai m\'{o}dszerek m\H{u}k\"{o}d\'{e}s\'{e}nek hat\'{e}konys\'{a}g\'{a}t, valamint  a m\'{a}s nyelveken v\'{e}gzett k\'{i}s\'{e}rletekkel \"{o}sszevetve k\'{e}pet kapjunk az esetlegesen felmer\"{u}l\H{o} magyar-specifikus kih\'{i}v\'{a}sokr\'{o}l: a rendelkez\'{e}sre \'{a}ll\'{o} er\H{o}forr\'{a}sokr\'{o}l, illetve a nyelvi jellegzetess\'{e}gekb\H{o}l ad\'{o}d\'{o} probl\'{e}m\'{a}kr\'{o}l.\\

A lexik\'{a}lis behelyettes\'{i}t\'{e}s jellemz\H{o}en k\'{e}t r\'{e}szfeladtra oszthat\'{o}. Az els\H{o} l\'{e}p\'{e}s a jel\"{o}ltek kinyer\'{e}se egy erre alkalmas jelent\'{e}st\'{a}rb\'{o}l (\'{a}ltal\'{a}ban wordnetb\H{o}l) vagy szinonima-adatb\'{a}zisb\'{o}l, illetve korpuszb\'{o}l disztrib\'{u}ci\'{o}s m\'{o}dszerekkel, pl. vektori\'{a}lis k\"{o}zels\'{e}g szerint. B\'{a}r sok kritika fogalmaz\'{o}dott meg a WordNet alkalmass\'{a}g\'{a}t illet\H{o}en (els\H{o}sorban jelent\'{e}segy\'{e}rtelm\H{u}s\'{i}t\'{e}si kontextusban, \cite{veronis03,idewilks06} illetve a magyarra \cite{heja09}, az angol nyelv\H{u} lexik\'{a}lis behelyettes\'{i}t\'{e}si verseny (SemEval 2007) sor\'{a}n a legjobbnak bizonyult m\'{o}dszerek m\'{e}gis mind t\'{a}maszkodnak a WordNet-re \cite{UNTSemeval,MELBSemeval}. A m\'{a}sodik l\'{e}p\'{e}s a jel\"{o}ltek rangsorol\'{a}sa aszerint, hogy melyik illeszkedik legjobban az adott sz\"{o}vegk\"{o}rnyezetbe. Ez a feladat k\"{o}zel \'{a}ll a jelent\'{e}segy\'{e}rtelm\H{u}s\'{i}t\'{e}shez, \'{a}m annot\'{a}lt szinonima-t\'{a}r hi\'{a}ny\'{a}ban nem t\'{a}maszkodhatunk fel\"{u}gyelt tan\'{i}t\'{a}si m\'{o}dszerekre. Lesk sz\'{o}t\'{a}ri defin\'{i}ci\'{o}kat \cite{lesk86}, Aguirre \'{e}s Rigau WordNet-alap\'{u} t\'{a}vols\'{a}gi m\'{e}rt\'{e}keket \cite{aguirrerigau96}, Carrol \'{e}s McCarthy szemantikai szelekci\'{o}s inform\'{a}ci\'{o}kat \cite{carroll00} haszn\'{a}l az egy\'{e}rtelm\H{u}s\'{i}t\'{e}shez. A disztrib\'{u}ci\'{o}s szemantik\'{a}ban haszn\'{a}lt vektori\'{a}lis sz\'{o}-reprezent\'{a}ci\'{o}k is alkalmasak r\'{a}, hogy szavak vagy nagyobb sz\"{o}vegegys\'{e}gek k\"{o}z\"{o}tti hasonl\'{o}s\'{a}gi m\'{e}rt\'{e}keket sz\'{a}m\'{i}tsunk bel\H{o}l\"{u}k. Egyes kutat\'{a}sok l\'{a}tens szemantikai dimenzi\'{o}kat alkalmaznak a sz\'{o}jelent\'{e}sek automatikus elk\"{u}l\"{o}n\'{i}t\'{e}s\'{e}re \'{e}s kontextusbeli egy\'{e}rtelm\H{u}s\'{i}t\'{e}s\'{e}re \cite{linpantel,vandecruys}. A szavak elosztott reprezent\'{a}ci\'{o}j\'{a}n (distributed lexical representations; word embedding) alapul\'{o} nyelvmodellek \cite{mikolov13} \'{a}ltal gener\'{a}lt vektori\'{a}lis reprezent\'{a}ci\'{o}k is alkalmasak arra, hogy rajtuk \'{e}rtelmezhet\H{o} k\"{o}zels\'{e}gi metrik\'{a}k alapj\'{a}n d\"{o}nts\"{u}nk a szavak szemantikai k\"{o}zels\'{e}g\'{e}r\H{o}l.
Ezek a m\'{o}dszerek t\"{o}bb SemEval versenyen - sz\'{o}hasonl\'{o}s\'{a}gi \'{e}s sz\'{o}anal\'{o}gi\'{a}s feladatok eset\'{e}ben - j\'{o}l teljes\'{i}tettek (Semeval 2012, 2014). A word2vec \cite{mikolov13} \'{e}s a GloVe \cite{pennington14} m\'{o}dszerek a szavakhoz  vagy tetsz\H{o}leges nagyobb egys\'{e}gekhez egy val\'{o}s vektort\'{e}rbeli vektort rendelnek, \'{u}gy, hogy az \'{i}gy l\'{e}trej\"{o}tt reprezent\'{a}ci\'{o}ra k\'{e}t tulajdons\'{a}g jellemz\H{o}: egyr\'{e}szt az egym\'{a}shoz k\"{o}zel es\H{o} szavak szemantikai illetve morfol\'{o}giai \'{e}rtelemben is k\"{o}zeliek, m\'{a}sr\'{e}szt a vektorok k\"{o}z\"{o}tti vektori\'{a}lis k\"{u}l\"{o}nbs\'{e}gek konzisztensek, \'{e}s egyik sz\'{o}p\'{a}rr\'{o}l a m\'{a}sikra \'{a}tvihet\H{o}k. Jellegzetes p\'{e}lda a sz\'{o}p\'{a}rok k\"{o}z\"{o}tt kinyerhet\H{o} anal\'{o}gi\'{a}s hasonl\'{o}s\'{a}gra: v(king) - v(queen) = v(man) - v(woman). Ez a k\'{e}t tulajdons\'{a}g indokolja a m\'{o}dszerek k\"{o}zvetlen haszn\'{a}lhat\'{o}s\'{a}g\'{a}t a sz\'{o}szemantikai feladatokban. A behelyettes\'{i}t\'{e}ses feladaton leg\'{u}jabban Ferret \cite{ferret14} v\'{e}gzett k\'{i}s\'{e}rletet francia nyelvre a word2vec \'{a}ltal gener\'{a}lt reprezent\'{a}ci\'{o} felhaszn\'{a}l\'{a}s\'{a}val. 

K\'{i}s\'{e}rlet\"{u}nkben l\'{e}trehozunk egy ilyen vektoros reprezent\'{a}ci\'{o}t magyar szavakra, \'{e}s ennek haszn\'{a}lhat\'{o}s\'{a}g\'{a}t mindk\'{e}t r\'{e}szfeladatra kipr\'{o}b\'{a}ljuk. M\'{a}sodsorban egy WordNet-alap\'{u} m\'{o}dszerrel pr\'{o}b\'{a}lkozunk \cite{gabor14}, mely a WordNet-beli lemm\'{a}kat, illetve a k\"{o}zt\"{u}k defini\'{a}lt hierarchikus kapcsolatokb\'{o}l sz\'{a}rmaz\'{o} inform\'{a}ci\'{o}t kombin\'{a}lja a disztrib\'{u}ci\'{o}s szemantika \'{e}s a dokumentumkategoriz\'{a}l\'{a}s ter\"{u}let\'{e}n haszn\'{a}lt elj\'{a}r\'{a}sokkal. A c\'{e}lsz\'{o} k\"{u}l\"{o}nb\"{o}z\H{o} jelent\'{e}seit, \'{e}s az ezekhez tartoz\'{o} lexikai egys\'{e}geket a WordNetb\H{o}l nyerj\"{u}k ki. A WordNet-jelent\'{e}sek klaszterez\'{e}se ut\'{a}n a jelent\'{e}seket k\"{o}r\"{u}lvev\H{o} relev\'{a}ns csom\'{o}pontok k\"{o}rbej\'{a}r\'{a}s\'{a}val tematikus kateg\'{o}ri\'{a}kat k\'{e}pez\"{u}nk, melyekhez ezut\'{a}n a korpuszb\'{o}l gy\H{u}jt\"{u}nk kateg\'{o}ria-specifikus kontextusokat. Az egy\'{e}rtelm\H{u}s\'{i}t\'{e}s sor\'{a}n a jel\"{o}ltek vektoros reprezent\'{a}ci\'{o}j\'{a}t vetj\"{u}k \"{o}ssze a kontextus szavaival. V\'{e}gul egy hibrid m\'{o}dszert is kipr\'{o}b\'{a}lunk, mely a WordNetb\H{o}l kinyert jel\"{o}lteket kiz\'{a}r\'{o}lag korpusz-alap\'{u} disztrib\'{u}ci\'{o}s inform\'{a}ci\'{o} felhaszn\'{a}l\'{a}s\'{a}val rangsorolja.




\section{Er\H{o}forr\'{a}sok}

\subsection{Magyar Nemzeti Sz\"{o}vegt\'{a}r}

A disztrib\'{u}ci\'{o}s inform\'{a}ci\'{o} kinyer\'{e}s\'{e}hez a Magyar Nemzeti Sz\"{o}vegt\'{a}r \cite{korpusz} (MNSZ, tov\'{a}bbiakban: korpusz) els\H{o}, kib\H{o}v\'{i}tett, elemzett v\'{a}ltozat\'{a}t haszn\'{a}ltuk \cite{korpusz,korpusz2}. A korpusz ezen v\'{a}ltozata 260 milli\'{o} sz\'{o}t tartalmaz, egy\'{e}rtelm\H{u}s\'{i}t\'{e}se az MNSZ egy\'{e}trelm\H{u}s\'{i}t\H{o} eszk\"{o}zl\'{a}nc seg\'{i}ts\'{e}g\'{e}vel \'{a}llt el\H{o} \cite{tagger}. A k\'{i}s\'{e}rleteinkhez a korpusz lemm\'{a}s\'{i}tott v\'{a}ltozat\'{a}t haszn\'{a}ltuk. 


\subsection{WordNet}


A wordnet olyan elektronikus lexik\'{a}lis szemantikai adatb\'{a}zis, melyben a nyelvi fogalmak h\'{a}l\'{o}zatba szervez\H{o}dnek. A fogalmakat szinonimahalmazok (synsetek), a k\"{o}z\"{o}tt\"{u}k l\'{e}v\H{o} kapcsolatokat szemantikai rel\'{a}ci\'{o}k (hipernima, meronima, antonima stb.) reprezent\'{a}lj\'{a}k. A WordNet alapegys\'{e}ge a szavakb\'{o}l \'{a}ll\'{o} szinonima-oszt\'{a}lyok, \'{u}gynevezett synsetek. \\
A magyar WordNet \cite{mihaltz08} mintegy 40.000 synsetet tartalmaz, melyek nagy r\'{e}sze meg van feleltetve ekvivalens angol WordNet synseteknek, \'{i}gy implicit m\'{o}don m\'{a}s nyelvek wordneteinek is.

\section{Jel\"{o}ltek kinyer\'{e}se a WordNetb\H{o}l}

A c\'{e}lsz\'{o} behelyettes\'{i}t\'{e}s\'{e}re sz\'{a}nt jel\"{o}lteket csoportosan nyerj\"{u}k ki a WordNet-b\H{o}l, ahol egy csoport a c\'{e}lsz\'{o} egy jelent\'{e}s\'{e}nek felel meg. M\'{o}dszer\'{e}nk c\'{e}lja egyfel\H{o}l, hogy a k\"{u}l\"{o}nb\"{o}z\H{o} (\'{e}s val\'{o}ban megk\"{u}l\"{o}nb\"{o}ztetend\H{o}) jelent\'{e}sek mindegyik\'{e}re tal\'{a}ljunk jel\"{o}ltet, m\'{a}sfel\H{o}l, hogy az \'{i}gy kapott jelent\'{e}sek k\'{e}s\H{o}bb hat\'{e}konyan felhaszn\'{a}lhat\'{o}k legyenek az egy\'{e}rtelm\H{u}s\'{i}t\'{e}sre. Fontos azonban megjegyezni, hogy nem k\"{o}zvetlen c\'{e}lunk a synsetek/jelent\'{e}sek k\"{o}z\"{u}l v\'{a}lasztani a mondatbeli behelyettes\'{i}t\'{e}s sor\'{a}n: a jelent\'{e}sek megk\'{e}l\"{o}nb\"{o}ztet\'{e}se csak a jel\"{o}lt-kinyer\'{e}s \'{e}s a jelent\'{e}s-specifikus kontextusok kiv\'{a}laszt\'{a}sa sor\'{a}n ker\"{u}l el\H{o}t\'{e}rbe.\\

A szinonim\'{a}k kinyer\'{e}s\'{e}nek els\H{o} l\'{e}p\'{e}sek\'{e}nt teh\'{a}t azonos\'{i}tjuk azokat a synseteket, melyek tartalmazz\'{a}k a c\'{e}lsz\'{o}t. Mivel a kor\'{a}bbi, angol nyelv\H{u} k\'{i}s\'{e}rletekhez hasonl\'{o}an \cite{UNTSemeval} a magyar WordNet eset\'{e}ben is el\H{o}fordul, hogy a c\'{e}lsz\'{o} az adott synsetb\H{o}l kinyerhet\H{o} egyetlen sz\'{o}, a keres\'{e}st ilyen esetekben kiterjesztett\"{u}k a hiperonim\'{a}kra is\footnote{A tesztadatok l\'{e}trehoz\'{a}sakor a SemEval 2007 feladathoz hasonl\'{o}an a magyar behelyettes\'{i}t\'{e}si feladatban is megengedt\'{e}k az \'{a}ltal\'{a}nosabb \'{e}rtelm\H{u} fogalommal val\'{o} helyettes\'{i}t\'{e}st.}.\\

K\"{o}zismert, hogy a WordNet nagyon r\'{e}szletes jelent\'{e}s-megk\"{u}l\"{o}nb\"{o}ztet\'{e}sekkel oper\'{a}l \cite{idewilks06}: sz\'{a}mos olyan megk\"{u}l\"{o}nb\"{o}ztet\'{e}st tartalmaz, mely az adott feladat kontextus\'{a}ban nem relev\'{a}ns, s\H{o}t kifejezetten megnehez\'{i}theti a jelent\'{e}segy\'{e}rtelm\H{u}s\'{i}t\'{e}st \cite{veronis03}. Mivel a WordNet szerkezete \"{o}nmag\'{a}ban nem felt\'{e}tlen\'{e}l ny\'{u}jt inform\'{a}ci\'{o}t a synsetek k\"{o}zti szemantikai t\'{a}vols\'{a}gr\'{o}l, \'{u}gy d\"{o}nt\"{o}tt\"{u}nk, hogy a synsetek lexik\'{a}lis tartalm\'{a}t felhaszn\'{a}lva pr\'{o}b\'{a}ljuk meg kisz\H{u}rni az irrelev\'{a}ns megk\"{u}l\"{o}nb\"{o}ztet\'{e}seket. Az egym\'{a}ssal megegyez\H{o} lexik\'{a}lis tartalm\'{u} synseteket teh\'{a}t \"{o}sszevontuk, csak\'{u}gy, mint azokat a synset-p\'{a}rokat, melyek k\"{o}z\"{u}l a kisebb synset r\'{e}szhalmaz\'{a}t alkotja a nagyobbnak. Az \"{o}sszevont synseteket a k\'{e}s\H{o}bbiekben nem k\"{u}l\"{o}nb\"{o}ztetj\"{u}k meg az eredeti synsetekt\H{o}l: valamennyit k\"{u}l\"{o}nb\"{o}z\H{o} jelent\'{e}sk\'{e}nt fogjuk kezelni.\\
 

\section{Vektor-alap\'{u} megk\"{o}zel\'{i}t\'{e}s}

A word embedding el\H{o}nye, hogy a szavak k\"{o}z\"{o}tt t\"{o}bbf\'{e}le szemantikai viszonyt k\'{e}pezhet\"{u}nk le egy val\'{o}s vektort\'{e}rben, \'{e}s ezeket egyszer\H{u} numerikus, line\'{a}ris m\'{o}dszerekkel t\'{a}rhatjuk fel. Megfigyelhet\H{o} p\'{e}ld\'{a}ul, hogy a c\'{e}lszavaknak egy adott jelent\'{e}shez tartoz\'{o} szinonim\'{a}i t\"{o}bbnyire egym\'{a}s k\"{o}zel\'{e}ben fordulnak el\H{o}. Emiatt a tulajdons\'{a}g miatt lehet\H{o}s\'{e}g van arra, hogy a lexik\'{a}lis behelyettes\'{i}t\'{e}si feladatot egy l\'{e}p\'{e}sben oldjuk meg: azokat a szavakat v\'{a}lasztjuk ki, amelyek tetsz\H{o}legesen v\'{a}lasztott szemantikai k\"{o}zels\'{e}gm\'{e}rt\'{e}k szerint a legk\"{o}zelebb esnek a c\'{e}lsz\'{o}hoz, hiszen ezek a legalkalmasak jel\"{o}ltnek. Ez a naiv megold\'{a}s egyszer\H{u} \'{e}s intuit\'{i}v, de nagyon j\'{o}l szerepel az oot-\'{e}rt\'{e}kel\'{e}sben, ez\'{e}rt baseline-nak tekinthetj\"{u}k. Mi a skal\'{a}rszorzatos megold\'{a}st implement\'{a}ltuk \emph{cosine} n\'{e}ven.\\

Megfigyelhetj\"{u}k azt is, hogy az egy adott szemantikai mez\H{o}be tartoz\'{o} szavak, szint\'{e}n k\"{o}zelebb esnek egym\'{a}shoz. Mivel gyakran el\H{o}fodul, hogy a c\'{e}lsz\'{o}val egy mez\H{o}be tartoz\'{o} tov\'{a}bbi szavak is vannak a kontextusban, ez\'{e}rt megk\'{i}s\'{e}rlhet\"{u}nk a jel\"{o}ltek k\"{o}z\"{u}l aszerint v\'{a}lasztani, hogy mennyire esnek k\"{o}zel a kontextusszavakhoz. Prec\'{i}zebben, minden jel\"{o}ltre kisz\'{a}molunk egy megfelel\H{o}s\'{e}gi m\'{e}rt\'{e}ket \'{u}gy, hogy \"{o}sszegezz\"{u}k a jel\"{o}lt \'{e}s minden egyes kontextussz\'{o} k\"{o}zels\'{e}gi m\'{e}rt\'{e}k\'{e}t, \'{e}s eszerint rendezve a legjobb jel\"{o}lteket adjuk vissza. \'{i}gy \cite{zweig12} megk\"{o}zel\'{i}t\'{e}s\'{e}t implement\'{a}ljuk. A k\"{o}zels\'{e}gi m\'{e}rt\'{e}k, t\"{o}bbek k\"{o}z\"{o}tt, lehet skal\'{a}rszorzat \'{e}s euklideszi t\'{a}vols\'{a}g, ezeknek az alkalmazott m\'{e}rt\'{e}k szerint bestcosinecontext \'{e}s bestl2context a neve a k\'{i}s\'{e}rlet\"{u}nkben. \\

Lehet\H{o}s\'{e}g\"{u}nk van arra is, hogy a c\'{e}lsz\'{o}hoz k\"{o}zel es\H{o} jel\"{o}ltek k\"{o}z\"{u}l azokat r\'{e}szes\'{i}ts\"{u}k el\H{o}nyben, amelyek a kontextushoz k\"{o}zelebb vannak, azaz amelyek a c\'{e}lsz\'{o} vektor\'{a}t\'{o}l a kontextus ered\H{o} vektora ir\'{a}ny\'{a}ba esnek. Ez a m\'{o}dszer t\"{o}bb param\'{e}tert is elfogad: a kontextus ered\H{o} vektor\'{a}nak kisz\'{a}m\'{i}t\'{a}sakor az egyes elemeket k\"{u}l\"{o}nf\'{e}lek\'{e}ppen s\'{u}lyozhatjuk, \'{e}s megv\'{a}laszthatjuk azt is, hogy milyen t\'{a}vols\'{a}gban keress\"{u}k a helyettes\'{i}t\H{o} szavakat az eredetit\H{o}l, azaz a kontextus \'{e}s a c\'{e}lsz\'{o} milyen line\'{a}ris kombin\'{a}ci\'{o}j\'{a}t sz\'{a}m\'{i}tjuk ki. A k\'{i}s\'{e}rletez\'{e}s sor\'{a}n azt tal\'{a}ltuk, hogy ha nagy s\'{u}ly adunk a kontextusnak, azaz t\'{a}volabb keresg\'{e}l\"{u}nk az eredeti c\'{e}lsz\'{o}t\'{o}l, r\'{a}tal\'{a}lhatunk egy-egy t\'{a}volabbi szinonim\'{a}ra is, de nagyobb sz\'{a}mban tal\'{a}lunk haszn\'{a}lhatatlan (nem behelyettes\'{i}thet\H{o}) szavakat is. Ezt a tulajdons\'{a}got a ki\'{e}rt\'{e}kel\H{o} adatok is igazolj\'{a}k: az \"{o}sszes k\'{i}s\'{e}rleti konfigur\'{a}ci\'{o}b\'{o}l az averagecontext szerepel az oot-ki\'{e}rt\'{e}kel\'{e}sben a legjobban (azaz nagyon j\'{o} jel\"{o}lteket is tal\'{a}l), de a best-\'{e}rt\'{e}kel\'{e}se nem j\'{o} (azaz sok m\'{a}s jel\"{o}ltje nem megfelel\H{o}).\\

Ugyanezeket a sz\'{a}m\'{i}t\'{a}sokat elv\'{e}gezhetj\"{u}k tetsz\"{o}leges szavakra, amelyeknek ismerj\"{u}k a vektort\'{e}rbeli reprezent\'{a}ci\'{o}it. Ez lehet\"{o}s\'{e}get ad egy egyszer\"{u} hibridiz\'{a}ci\'{o}ra: m\'{a}s m\'{o}dszerek \'{a}ltal gener\'{a}lt jel\"{o}lteket tudunk a fent le\'{i}rt m\'{o}dszerekkel ki\'{e}rt\'{e}kelni \'{e}s sorbarendezni. A ki\'{e}rt\'{e}kel\'{e}sben l\'{a}that\'{o}, hogy a hybrid bestcosinecontext konfigur\'{a}ci\'{o} \"{o}tv\"{o}zi a k\"{u}l\"{o}nb\"{o}z\"{o} megk\"{o}zel\'{i}t\'{e}sek el\"{o}nyeit, \'{e}s \"{o}sszess\'{e}g\'{e}ben a legjobb eredm\'{e}nyeket adja. Ebben a konfigur\'{a}ci\'{o}ban a WordNetb\"{o}l kinyert jel\"{o}lteket rangsoroltuk a bestcosinecontext \'{e}rt\'{e}k alapj\'{a}n.

\section{WordNet-alap\'{u} megk\"{o}zel\'{i}t\'{e}s}

A WordNet-alap\'{u} megk\"{o}zel\'{i}t\'{e}s motiv\'{a}ci\'{o}ja a WordNet szerkezet\'{e}ben rejl\H{o} inform\'{a}ci\'{o} kihaszn\'{a}l\'{a}sa \'{e}s \"{o}tv\"{o}z\'{e}se a korpusz-alap\'{u} megk\"{o}zel\'{i}t\'{e}s el\H{o}nyeivel. A folyamat els\H{o} l\'{e}p\'{e}s\'{e}ben a jel\"{o}lteket a c\'{e}lsz\'{o}t tartalmaz\'{o} synsetekb\H{o}l nyert\"{u}k ki. Ezut\'{a}n a c\'{e}lsz\'{o} synsetjeit csoportos\'{i}tjuk, \'{e}s az \'{i}gy kapott jelent\'{e}sekhez a synsetek tartalm\'{a}t felhaszn\'{a}lva keres\'{e}nk olyan kontextusokat a korpuszban, melyek az adott jelent\'{e}sre specifikusak, \'{e}s \'{i}gy megk\"{u}l\"{o}nb\"{o}ztet\H{o} er\H{o}vel b\'{i}rnak az egy\'{e}rtelm\H{u}s\'{i}t\'{e}s sor\'{a}n. Ezek a kontextusok fogj\'{a}k k\'{e}pezni a c\'{e}lsz\'{o} specifikus egy\'{e}rtelm\H{u}s\'{i}t\H{o} vektoter\'{e}t, melyen valamennyi jel\"{o}ltet elhelyezz\'{e}k. V\'{e}g\"{u}l az utols\'{o} l\'{e}p\'{e}sben a mondat szavait vetj\'{e}k \"{o}ssze a jel\"{o}lteknek az egy\'{e}rtelm\H{u}s\'{i}t\H{o} kontextusokra felvett \'{e}rt\'{e}keivel, \'{e}s eszerint rangsoroljuk \H{o}ket.


\subsection{Jelent\'{e}sek megk\"{u}l\"{o}nb\"{o}ztet\'{e}se}

A WordNet nemcsak azt teszi lehet\H{o}v\'{e}, hogy relev\'{a}ns behelyettes\'{i}t\'{e}si jel\"{o}lteket gener\'{a}ljunk. A munkafolyamat k\"{o}vetkez\H{o} l\'{e}p\'{e}seinek c\'{e}lja, hogy a WordNet szerkezet\'{e}t \'{e}s lexik\'{a}lis tartalm\'{a}t kihaszn\'{a}lva \"{o}sszegy\H{u}jts\"{u}k azokat a kontextusokat, melyek v\'{e}lhet\H{o}en relev\'{a}ns inform\'{a}ci\'{o}t hordoznak a jelent\'{e}sek, \'{e}s ezen kereszt\"{u}l a behelyettes\'{i}t\'{e}si jel\"{o}ltek kontextusbeli egy\'{e}rtelm\H{u}s\'{i}t\'{e}s\'{e}hez. Felt\'{e}telez\'{e}s\"{u}nk szerint e megk\"{o}zel\'{i}t\'{e}s egyik el\H{o}nye lehet, hogy egy tetsz\H{o}leges korpuszban a szavak ritk\'{a}bb jelent\'{e}sei meglehet\H{o}sen alulreprezent\'{a}ltak, ez\'{e}rt a tiszt\'{a}n disztrib\'{u}ci\'{o}s, illetve nyelvmodell alap\'{u} egy\'{e}rtelm\H{u}s\'{i}t\'{e}si m\'{o}dszerek ezen jelent\'{e}sek el\H{o}fordul\'{a}sait nehezen tudj\'{a}k azonos\'{i}tani. Ha azonban rendelkez\'{e}s\"{u}nkre \'{a}ll egy lista a sz\'{o} lehets\'{e}ges jelent\'{e}seir\H{o}l, valamint a k\"{u}l\"{o}nb\"{o}z\H{o} jelent\'{e}sekkel kapcsolatba hozhat\'{o} szavakr\'{o}l - melyek a wordnet-hierarchi\'{a}b\'{o}l k\"{o}nnyed\'{e}n kinyerhet\H{o}k - lehet\H{o}s\'{e}g\"{u}nk ny\'{i}lik arra, hogy olyan kontextusokat is figyelembe vegy\"{u}nk az egy\'{e}rtelm\H{u}s\'{i}t\'{e}s sor\'{a}n, melyek egy\'{e}bk\'{e}nt a c\'{e}lsz\'{o}val, illetve a behelyettes\'{i}t\'{e}si jel\"{o}ltjeivel nem, vagy csak kev\'{e}sszer fordulnak el\H{o} a korpuszban. A kor\'{a}bbiakban bemutatott tiszt\'{a}n disztrib\'{u}ci\'{o}s alap\'{u} megk\"{o}zel\'{I}t\'{e}ssel szemben teh\'{a}t a m\'{a}sodik k\'{i}s\'{e}rlet c\'{e}lja megvizsg\'{a}lni, hat\'{e}kony lehet-e egy k\"{u}ls\H{o} er\H{o}forr\'{a}s bevon\'{a}sa az egy\'{e}rtelm\H{u}s\'{i}t\'{e}si folyamatba. \\

A jelent\'{e}sek megk\"{u}l\"{o}nb\"{o}ztet\'{e}s\'{e}hez az el\H{o}z\H{o} pontban bemutatott synset-klaszterekb\H{o}l indulunk ki.  C\'{e}lunk, hogy minden jelent\'{e}st megfelel\H{o} mennyis\'{e}g\H{u} lexik\'{a}lis elemmel tudjunk reprezent\'{a}lni. Ezen lexik\'{a}lis elemek r\'{e}szben az el\H{o}z\H{o} pontban kinyert behelyettes\'{i}t\'{e}si jel\"{o}ltek (szinonim\'{a}k, illetve ezek hi\'{a}ny\'{a}ban hiperonim\'{a}k). A k\"{o}vetkez\H{o} l\'{e}p\'{e}sben azokhoz a jelent\'{e}sekhez, melyek h\'{a}romn\'{a}l kevesebb lexik\'{a}lis elemmel hozhat\'{o}k kapcsolatba, \'{u}jabb szavakat kerest\"{u}nk a WordNet k\"{o}rnyez\H{o} synsetjeiben: minden olyan synsetben, mely k\"{o}zvetlen\"{u}l hiperonim, "category\_domain", illetve "mero\_part" rel\'{a}ci\'{o}ban \'{a}ll az adott synsettel (\"{o}sszevon\'{a}s eset\'{e}n a kiindul\'{o} synsetek valamelyik\'{e}vel). Ezen kapcsol\'{o}d\'{o} synsetek lexik\'{a}lis tartalm\'{a}t a jelent\'{e}shez csatoltuk \footnote{Fontos megjegyezni, hogy az \'{i}gy kinyert \'{u}j szavak m\'{a}r nem sz\'{a}m\'{i}tanak behelyettes\'{i}t\'{e}si jel\"{o}ltnek, kiz\'{a}r\'{o}lag a jelent\'{e}sek jellemz\H{o} kontextusainak felt\'{e}rk\'{e}pez\'{e}s\'{e}hez haszn\'{a}ltuk \H{o}ket.}.\\

A behelyettes\'{i}t\'{e}si jel\"{o}lteket a k\'{e}s\H{o}bbiekben \'{u}gy k\'{i}v\'{a}njuk kiv\'{a}lasztani, hogy a mondatbeli kontextust \"{o}sszevetj\"{u}k az egyes jelent\'{e}sekre jellemz\H{o} kontextusokkal. Ehhez l\'{e}tre kell hoznunk egy olyan "egy\'{e}rtelm\H{u}s\'{i}t\'{e}si vektorteret", mely a c\'{e}lsz\'{o} \"{o}sszes k\"{u}l\"{o}nb\"{o}z\H{o} jelent\'{e}s\'{e}nek legink\'{a}bb jellemz\H{o} kontextusait tartalmazza, \'{e}s a lehet\H{o} legkev\'{e}sb\'{e} favoriz\'{a}lja a gyakori jelent\'{e}seket. 
A c\'{e}lsz\'{o} jelent\'{e}seihez t\'{a}rs\'{i}tott valamennyi sz\'{o} minden el\H{o}fordul\'{a}s\'{a}t figyelembe v\'{e}ve kinyert\"{u}k a korpuszb\'{o}l a szavak kontextusait. Szintaktikai elemz\'{e}s h\'{i}j\'{a}n kontextusk\'{e}nt kezelt\"{u}nk minden, a sz\'{o} k\"{o}rnyezet\'{e}ben el\H{o}fordul\'{o} lemm\'{a}t, kett\H{o}, illetve \"{o}t sz\'{o}b\'{o}l \'{a}ll\'{o} kontextus-ablakok haszn\'{a}lat\'{a}val (ezek a sz\'{o}ablakok bizonyultak a legeredm\'{e}nyesebbnek \cite{baroni14} r\'{e}szletes \"{o}sszehasonl\'{i}t\'{o} vizsg\'{a}lat\'{a}ban). Ezut\'{a}n minden \emph{S} synset-klaszterre kikerest\"{u}k azokat a \emph{w} szavakat, melyek az adott synset-klaszter \emph{s} szavaira legink\'{a}bb jellemz\H{o}ek az al\'{a}bbi k\'{e}plet szerint :
 
\begin{equation}
spec_{w,S} = \sum_{s \in S} weight_{s,w}
\end{equation}

ahol a \emph{weight} s\'{u}ly a PMI (Pointwise Mutual Information) egy normaliz\'{a}lt v\'{a}ltozata. A PMI kollok\'{a}ci\'{o}k \'{e}s jellemz\H{o} kontextusok kinyer\'{e}s\'{e}nek bevett m\'{o}dja:

\begin{equation}
PMI_{w_1,w_2} = log \frac{p(w_1, w_2)}{p(w_1) p(w_2)}
\end{equation}

A PMI h\'{a}tr\'{a}nya azonban, hogy er\H{o}sen favoriz\'{a}lja a ritka kontextusokat. Eset\"{u}nkben ez az egy\'{e}rtelm\H{u}s\'{i}t\'{e}si feladatot megnehez\'{i}ti, ez\'{e}rt k\'{e}t normaliz\'{a}l\'{a}si elj\'{a}r\'{a}st is kipr\'{o}b\'{a}ltunk, hogy kiv\'{e}dj\"{u}k a ritka kontextusok fel\"{u}lreprezent\'{a}l\'{a}s\'{a}t \cite{bouma,thanospoulosetal}:

\begin{equation}
NPMI_{w_1,w_2} = log \frac{p(w_1, w_2)}{p(w_1) p(w_2)} / - log ~~ p (w_1, w_2)
\end{equation}

\begin{equation}
squaredPMI_{w_1,w_2} = log \frac{p^{2}(w_1, w_2)}{p(w_1) p(w_2)}
\end{equation}
 


A jelent\'{e}s-specifikus kontextusokat a fenti \'{e}rt\'{e}kek szerint rangsoroltuk. A c\'{e}lsz\'{o} egy\'{e}rtelm\H{u}s\'{i}t\H{o} vektotere a c\'{e}lsz\'{o} jelent\'{e}seire jellemz\H{o} kontextusok uni\'{o}j\'{a}b\'{o}l \'{a}ll el\H{o}: jelent\'{e}senk\'{e}nt a legjellemz\H{o}bb 200, illetve 500 kontextust tartottuk meg. El\H{o}fordulhat, hogy egy kontextus t\"{o}bb jelent\'{e}sre is jellemz\H{o}, ezt is hasznos inform\'{a}ci\'{o}nak tekintett\"{u}k. A vektorterek m\'{e}rete \'{i}gy a c\'{e}lsz\'{o} jelent\'{e}sei sz\'{a}m\'{a}nak, \'{e}s a jelent\'{e}sek \'{a}tfed\'{e}s\'{e}nek f\"{u}ggv\'{e}ny\'{e}ben v\'{a}ltoz\'{o}. \\

Mivel feladatunk nem k\"{o}zvetlen\"{u}l a jelent\'{e}segy\'{e}rtelm\H{u}s\'{i}t\'{e}s, hanem a legmegfelel\H{o}bb jelent\'{e}s kiv\'{a}laszt\'{a}sa, ez\'{e}rt a c\'{e}lsz\'{o}hoz tartoz\'{o} \"{o}sszes behelyetts\'{i}t\'{e}si jel\"{o}ltet elhelyezz\"{u}k a fenti vektort\'{e}rben. Ehhez h\'{a}rom k\"{u}l\"{o}nb\"{o}z\H{o} reprezent\'{a}ci\'{o}t haszn\'{a}ltunk: a c\'{e}lsz\'{o} \'{e}s a kontextus egy\"{u}ttes el\H{o}fodul\'{a}sainak sz\'{a}m\'{a}t ($ freq(c,w) $),  a c\'{e}lsz\'{o} kontextus melletti relat\'{i}v gyakoris\'{a}g\'{a}t:

\begin{equation}
\frac{freq(c,w)}{freq(c)}
\end{equation}

illetve a relat\'{i}v gyakoris\'{a}g normaliz\'{a}lt \'{e}rt\'{e}k\'{e}t:
 
 \begin{equation}
 \frac{\frac{freq(c,w)}{freq(c)} - \mu}{\sqrt{\frac{\sigma^2}{N}}} 
 \end{equation}
 	
Az egy\"{u}ttes el\H{o}fordul\'{a}sokat ugyanolyan kontextus-ablakot haszn\'{a}lva sz\'{a}moltuk, ahogyan az egy\'{e}rtelm\H{u}s\'{i}t\H{o} vektotereket el\H{o}\'{a}ll\'{i}tottuk.



\subsection{Egy\'{e}rtelm\H{u}s\'{i}t\'{e}s}


A mondatbeli kontextusba illeszked\H{o} jel\"{o}lt kiv\'{a}laszt\'{a}sakor a jel\"{o}lteknek az egy\'{e}rtelm\H{u}s\'{i}t\H{o} vektort\'{e}rben alkotott reprezent\'{a}ci\'{o}j\'{a}t vetj\"{u}k \"{o}ssze a mondat szavaival. Ehhez el\H{o}sz\"{o}r is lemmatiz\'{a}ltuk a teszmondatokat az MNSZ egy\'{e}rtelm\H{u}s\'{i}t\H{o} eszk\"{o}zl\'{a}nc \cite{tagger} seg\'{i}ts\'{e}g\'{e}vel. A mondat \emph{p} vektora \'{u}gy \'{a}ll el\H{o}, hogy a mondat \emph{i} szavait is r\'{a}k\'{e}pezz\"{u}k a c\'{e}lsz\'{o} egy\'{e}rtelm\H{u}s\'{i}t\H{o} vektorter\'{e}re egy karakterisztikus f\"{u}ggv\'{e}nnyel:

\[
   p_i = 
\begin{cases}
    1, ~ \text{ha i el\H{o}fordul a mondatban} \\
    0,  ~ \text{egy\'{e}bk\'{e}nt}
\end{cases}
\]

A jel\"{o}lteket eztu\'{a}n a \emph{p} mondat-vektor \'{e}s a jel\"{o}lt \emph{c} egy\'{e}rtelm\H{u}s\'{i}t\H{o} vektora k\"{o}zti kompatibilit\'{a}s szerint rangsoroljuk, melyet a k\"{o}vetkez\H{o} k\'{e}plet szerint sz\'{a}molunk ki:


\begin{equation}
compatibility (c,p) = c \cdot p = \sum_{i=1}^{n} c_i \times p_i
\end{equation}

A mondat szavai k\"{o}z\"{u}l teh\'{a}t csak azokat vessz\"{u}k figyelembe, melyek a c\'{e}lsz\'{o} valamelyik jelent\'{e}s\'{e}hez specifikus kontextusk\'{e}nt lettek t\'{a}rs\'{i}tva, \'{e}s azzal a s\'{u}llyal esnek latba, amit az adott jel\"{o}lth\"{o}z rendelt\"{u}nk annak korpuszbeli el\H{o}fordul\'{a}sai alapj\'{a}n.



\section{Ki\'{e}rt\'{e}kel\'{e}s \'{e}s perspekt\'{i}v\'{a}k}

Az eredm\'{e}nyek ki\'{e}rt\'{e}kel\'{e}s\'{e}hez haszn\'{a}lt adatok elk\'{e}sz\'{i}t\'{e}sekor a McCarthy \'{e}s Navigli (Semeval 2007), illetve a Fabre \'{e}s tsai (SemDis 2014) \'{a}ltal k\"{o}vetett m\'{o}dszert vett\"{u}k alapul. T\'{i}z polisz\'{e}m f\H{o}nevet v\'{a}lasztottunk a ki\'{e}rt\'{e}kel\'{e}shez, melyek minden jelent\'{e}s\"{u}kben rendelkeznek egytag\'{u} szinonim\'{a}val. Felt\'{e}tel volt tov\'{a}bb\'{a}, hogy maga a f\H{o}n\'{e}v, valamint szinonim\'{a}i (jelent\'{e}senk\'{e}nt legal\'{a}bb egy) is kell\H{o} m\'{e}rt\'{e}kben reprezent\'{a}lva legyenek a rendelkez\'{e}sre \'{a}ll\'{o} korpuszban, hiszen els\H{o}sorban a korpusz-alap\'{u} disztrib\'{u}ci\'{o}s komponensek teljes\'{i}tm\'{e}ny\'{e}t szeretn\'{e}nk ki\'{e}rt\'{e}kelni. Minden c\'{e}lsz\'{o}hoz 10- 10 p\'{e}ldamondatot kerest\"{u}nk oly m\'{o}don, hogy minden sz\'{o}nak minden jelent\'{e}se reprezent\'{a}lva legyen. A c\'{e}lsz\'{o} mondatbeli el\H	{o}fordul\'{a}saihoz ezut\'{a}n legal\'{a}bb 3-3 annot\'{a}tor javasolt mondatonk\'{e}nt legfeljebb n\'{e}gy behelyettes\'{i}thet\H{o} lexikai elemet. A rendszer \'{a}ltal javasolt megold\'{a}sokat ezekkel a ki\'{e}rt\'{e}kel\'{e}si adatokkal vethetj\"{u}k \"{o}ssze, figyelembe v\'{e}ve azt is, hogy a rendszer megold\'{a}sa h\'{a}ny annot\'{a}tor javaslatai k\"{o}z\"{o}tt szerepel. \\

Hasonl\'{o}an a kor\'{a}bbi lexik\'{a}lis behelyettes\'{i}t\'{e}si feladatokhoz, k\'{e}tf\'{e}le m\'{e}rt\'{e}ket haszn\'{a}ltunk a gold sztenderddel val\'{o} \"{o}sszevet\'{e}shez \cite{mccarthynavigli10} : a 'best' m\'{e}rt\'{e}k a rendszer els\H{o}nek rangsorolt javaslat\'{a}t veszi figyelembe, m\'{i}g az 'oot' (\emph{out of ten}) azt m\'{e}ri, hogy az els\H{o} t\'{i}z javaslat k\"{o}z\"{o}tt h\'{a}ny j\'{o} jel\"{o}lt szerepel, a sorrendre val\'{o} tekintet n\'{e}lk\"{u}l. Mivel a WordNet-alap\'{u} m\'{o}dszerek gyakran enn\'{e}l kevesebb (\'{e}s csak nagyon ritk\'{a}n t\"{o}bb) jel\"{o}ltb\H{o}l indulnak ki, ez\'{e}rt ebben az esetben az oot \'{e}rt\'{e}k ink\'{a}bb a WordNet mint forr\'{a}s lefedetts\'{e}g\'{e}nek indik\'{a}tora. Az egy mondatra adott best \'{e}rt\'{e}k azt mutatja meg, hogy a rendszer \'{a}ltal javasolt legjobb jel\"{o}lt h\'{a}nyszor szerepel az annot\'{a}torok megold\'{a}sai k\"{o}z\"{o}tt (min\'{e}l t\"{o}bben javasolt\'{a}k, ann\'{a}l val\'{o}sz\'{i}n\H{u}bb, hogy er\H{o}s jel\"{o}lt), elosztva az annot\'{a}torok \'{a}ltal javasolt \"{o}sszes megold\'{a}s sz\'{a}m\'{a}val. A rendszer best mutat\'{o}ja az \"{o}sszes mondatra kapott best \'{e}rt\'{e}kek \'{a}tlaga. Az oot \'{e}rt\'{e}k sz\'{a}m\'{i}t\'{a}sakor a rendszer \'{a}ltal javasolt els\H{o} t\'{i}z jel\"{o}lt pontsz\'{a}mait (azaz szint\'{e}n az egyes annot\'{a}torok\'{e}val egyez\H{o} javaslatok pontsz\'{a}m\'{a}t) osztjuk el az \"{o}sszes annot\'{a}tor \'{a}ltal tett javaslatok sz\'{a}m\'{a}val.\\

\begin{table}
\centering
\begin{tabular}{lrr}
{\bf M\'{o}dszer} & {\bf BEST} & {\bf OOT}\\               
\tabularnewline
\midrule
\bf{veconly bestcosinecontext } & \bf{0.06702} &  0.267739 \\
veconly bestl2context & 0.05913 &  0.25895 \\
veconly averagecontext & 0.02997 & \bf{0.29371} \\
veconly cosine & 0.02806 &  0.28349
\tabularnewline
\midrule
\bf{wnet.lemma2.size200.NPMI.rawcount.txt }  & \bf{0.11064}  & \bf{0.23881} \\
wnet.lemma5.size200.NPMI.rawcount.txt   & 0.09560 & 0.23881 \\
wnet.lemma2.size200.NPMI.relfreqnorm.txt  & 0.09451 & 0.22743 \\
wnet.lemma2.size500.NPMI.rawcount.txt    & 0.09423 & 0.23881 \\
wnet.lemma5.size500.NPMI.rawcount.txt    & 0.08731 & 0.23881 \\ 
wnet.lemma5.size200.NPMI.relfreqnorm.txt & 0.08717 & 0.22410 \\
\tabularnewline
\midrule
\bf{hybrid bestcosinecontext} & \bf{0.11029} & \bf{0.24003} \\
hybrid bestl2context & 0.07988 & 0.23741 \\
\bottomrule
\end{tabular}
\caption{Eredm\'{e}nyek m\'{o}dszerenk\'{e}nti bont\'{a}sban}
\end{table}

Eddigi munk\'{a}nk term\'{e}szetes folytat\'{a}sa lehet az MNSZ2 teljes anyag\'{a}nak felhaszn\'{a}l\'{a}sa a disztrib\'{u}ci\'{o}s modellek sz\'{a}m\'{i}t\'{a}sakor. Ig\'{e}retes lehet\"{o}s\'{e}g a hibrid megk\"{o}zel\'{i}t\'{e}s tov\'{a}bbi kombin\'{a}ci\'{o}inak ki\'{e}rt\'{e}kel\'{e}se, tov\'{a}bb\'{a} az optim\'{a}lis vektoros reprezent\'{a}ci\'{o} megkeres\'{e}se a param\'{e}terek finomhangol\'{a}s\'{a}val. Tov\'{a}bbi annot\'{a}tori munk\'{a}val lehets\'{e}ges lenne a tesztanyag \"{o}sszek\"{o}t\'{e}se a m\'{a}r el\'{e}rhet\"{o} magyar jelent\'{e}segy\'{e}rtelm\H{u}s\'{i}t\"{o} korpusszal (hunwsd), illetve az \'{a}ltalunk gy\H{u}jt\"{o}tt goldstandard annot\'{a}l\'{a}sa jelent\'{e}sekkel.\\

A k\'{i}s\'{e}rletek sor\'{a}n k\'{e}zi \'{e}s g\'{e}pi munk\'{a}val l\'{e}trehozott adatokat szabadon el\'{e}rhet\"{o}v\'{e} tessz\"{u}k.
 

\section{K\"osz\"onetnyilv\'an\'\i t\'as}

Ez\'{u}ton k\"{o}sz\"{o}nj\"{u}k Oravecz Csab\'{a}nak, az MTA Nyelvtudom\'anyi Int\'ezet munkat\'{a}rs\'{a}nak az MNSZ-egy\'{e}rtelm\H{u}s\'{I}t\H{o} eszk\"{o}zl\'{a}nc rendelkez\'{e}s\"{u}nkre bocs\'{a}t\'{a}s\'{a}t \'{e}s a haszn\'{a}lat\'{a}ban ny\'{u}jtott seg\'{i}ts\'{e}g\'{e}t.
%
% ---- Bibliography ----
%
\begin{thebibliography}{15}
%
\bibitem{aguirrerigau96}
Aguirre E., Rigau G.:
Word Sense Disambiguation using Conceptual Density.
In Proceedings of COLING'96, 16--22. (1996)
%
\bibitem{baroni14}
Baroni M., Dinu G., Kruszewski G.:
Don't count, predict! A systematic comparison of context-counting vs. context-predicting semantic vectors. 
In Proceedings of the ACL Conference. (2014)
%
\bibitem{bouma}
Bouma G.:
Normalized (Pointwise) Mutual Information in Collocation Extraction.
In: From Form to Meaning: Processing Texts Automatically, Proceedings of the Biennial GSCL Conference. 31--40 (2009),
%
\bibitem{carroll00}
Carroll J., McCarthy D.:
Word Sense Disambiguation Using Automatically Acquired Verbal Preferences.
In Computers and the Humanitie. vol.34 109--114. (2000)%
\bibitem{fabre14}
Fabre C., Hathout N., Ho-Dac L., Morlane-Hond\`{e}re F., Muller P., Sajous F., Tanguy L., Van de Cruys T.:
Pr\'{e}sentation de l'atelier SemDis 2014 : S\'{e}mantique distributionnelle pour la substitution lexicale et l'exploration de corpus sp\'{e}cialis\'{e}s.
In Proceedings of the TALN 2014 Conference, Marseille, France. (2014)
%
\bibitem{ferret14}
Ferret O.:
Using a generic neural model for lexical substitution (Utiliser un mod{\`e}le neuronal g{\'e}n{\'e}rique pour la substitution lexicale)
In TALN-RECITAL 2014 Workshop SemDis 2014 : Enjeux actuels de la s{\'e}mantique distributionnelle, 218--227 (2014)
%
\bibitem{gabor14}
G\'{a}bor K.:
The WoDiS System - WOlf and DIStributions for Lexical Substitution (Le syst{\`e}me WoDiS - WOLF et DIStributions pour la substitution lexicale)
In TALN-RECITAL 2014 Workshop SemDis 2014 : Enjeux actuels de la s{\'e}mantique distributionnelle, 228--237 (2014)
%
\bibitem{UNTSemeval}
Hassan S., Csomai A., Banea C., Sinha R., Mihalcea R.:
Unt : Subfinder : Combining knowledge sources for automatic lexical substitution. 
In Proceedings of the Fourth International Workshop on Semantic Evaluations (SemEval-2007), Prague, Czech Republic : Association for Computational Linguistics. (2007)
%
\bibitem{heja09}
H\'{e}ja E., Kuti J., Sass B.
Jelent\'{e}segy\'{e}rtelm\H{u}s\'{i}t\'{e}s - egy\'{e}rtelm\H{u} jelent\'{e}s?
In: MSZNY2009, VI. Magyar Sz\'{a}m\'{i}t\'{o}g\'{e}pes Nyelv\'{e}szeti Konferencia, SZTE, Szeged. 348--352. (2009) 
%
\bibitem{idewilks06} 
Ide N., Wilks Y.:
Making sense about sense. 
In Word Sense Disambiguation : Algorithms and Applications, vol. 33 of Text, Speech and Language Technology, 47--74. Dordrecht, The Netherlands : Springer. (2006)
%
\bibitem{lesk86}
Lesk M.:
Automatic Sense Disambiguation Using Machine Readable Dictionaries: How to tell a pine cone from a ice cream cone.
In Proceedings of SIGDOC-1986 (1986)
%
\bibitem{linpantel}
Lin D., Pantel P.:
Concept discovery from text.
In Proceedings of the 19th International Conference on Computational Linguistics (COLING) (2002)
%
\bibitem{MELBSemeval}
Martinez D., Kim S. N., Baldwin T.:
Melb-mkb : Lexical substitution system based on relatives in context. 
In Proceedings of the Fourth International Workshop on Semantic Evaluations (SemEval-2007), Prague, CzechRepublic : Association for Computational Linguistics (2007)
%
\bibitem{mccarthynavigli10}
McCarthy D., Navigli R.:
The English Lexical Substitution Task.
in Language Resources and Evaluation, 43(2). 139--159 (2009).
%
\bibitem{mihaltz08}
Mih\'{a}ltz M., Hatvani Cs., Kuti J., Szarvas Gy., Csirik J., Pr\'{o}sz\'{e}ky G., V\'{a}radi T.:
Methods and Results of the Hungarian WordNet Project.
 In: Proceedings of The Fourth Global WordNet Conference, Szeged, Hungary, 311--321. (2008)
%
\bibitem{mikolov13}
Mikolov T., Sutskever I., Chen K., Corrado G., Dean J.:
Distributed Representations of Words and Phrases and their Compositionality.
In Proceedings of NIPS. (2013) 
%
\bibitem{tagger}
Oravecz Cs., Dienes P.:
Efficient Stochastic Part-of-Speech tagging for Hungarian.
In Proceedings of the Third International Conference on Language Resources and Evaluation, pages 710-717. (2002)
%
\bibitem{korpusz2}
Oravecz Cs., V\'{a}radi T., Sass B.:
The Hungarian Gigaword Corpus.
In: Proceedings of the International Conference on Language Resources and Evaluation (LREC) European Language Resources Association. (2014).
%
\bibitem{pado07}
Pad\'{o} U.:
The Integration of Syntax and Semantic Plausibility in a Wide-Coverage Model of Sentence Processing.
Dissertation, Saarland University, Saarbucken. (2007)
%
\bibitem{pennington14}
Pennington J., Socher R., Manning C.:
Glove: global vectors for word representation
Empirical Methods in Natural Language Processing (EMNLP), 2014 (to appear)
%
\bibitem{rubens65}
Rubenstein H., Goodenough J.:
Contextual correlates of synonymy.
in Communications of the ACM, 8(10), 627--633 (1965).
%
\bibitem{thanospoulosetal}
Thanopoulos A., Fakotakis N., Kokkinakis G.:
Comparative Evaluation of Collocation Extraction Metrics.
In Proceedings of the Third International Conference on Language Resources and Evaluation. (2002)
%
\bibitem{vandecruys}
van de Cruys T., Poibeau T., Korhonen A.:
Latent vector weighting for word meaning in context.
In Proceedings of the EMNLP 2011 Conference, 1012--1022 (2011)
%
\bibitem{korpusz}
V\'{a}radi T.:
The Hungarian National Corpus.
In: Proceedings of the International Conference on Language Resources and Evaluation (LREC) European Language Resources Association.385--389. (2002)
%
\bibitem{veronis03}
V\'{e}ronis J.:
Sense tagging : does it make sense ? 
In Corpus Linguistics by the Lune : a festschrift for Geoffrey Leech. Frankfurt : Peter Lang (2003)
%
\bibitem{zweig12}
Zweig G., Platt J. C., Meek C., Burges C. J., Yessenalina A., Liu Q.:
Computational approaches to sentence completion.
In 50th Annual Meeting of the Association for Computational Linguistics (ACL?12), p. 601?610, Jeju Island, Korea. (2012)

\end{thebibliography}
\end{document}
